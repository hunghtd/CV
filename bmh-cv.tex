\documentclass[11pt]{article}

% include data on fonts

\usepackage{ucs}
\usepackage[utf8x]{inputenc}

\usepackage[T1]{fontenc}
\usepackage{textcomp}
\renewcommand{\rmdefault}{ugm}
\renewcommand{\sfdefault}{phv}
\usepackage[garamond]{mathdesign}

\usepackage[letterpaper,left=1.85in,right=1.1in,top=1.1in,bottom=1.1in]{geometry}

% Metadata -- alter as neded

\def\myauthor{Benjamin Mako Hill}
\def\mytitle{Curriculum Vitæ}
\def\mycopyright{\myauthor}
\def\myaffiliation{Massachusetts Institute of Technology}
\def\myaddress{}
\def\myemail{mako@mit.edu}
\def\myweb{http://mako.cc}
\def\myphone{(+1) 206-409-7191}
\def\myfax{(+1) 815-361-75092}

% Git version tracking 
\input{vc}

% Required style files
\usepackage{url,fancyhdr}

% color for the links 
\usepackage[usenames,dvipsnames]{color}

% import and customize urls
\usepackage[usenames,dvipsnames]{color}
\usepackage[breaklinks]{hyperref}

\hypersetup{colorlinks=true, linkcolor=Blue, citecolor=Black, filecolor=Blue,
    urlcolor=Blue, unicode=true}

% customize the titles so that they appear in the right margin
\usepackage{titlesec}


\setlength{\marginparwidth}{4.2in}
\setlength{\titlewidth}{2.2in}

\titleformat{\section}[leftmargin]
{\normalfont 
\sffamily\bfseries\filleft}
{}{0pt}{\color{BrickRed}}
\titlespacing{\section}
{1.0in}{1.5ex plus .1ex minus .2ex}{1pc}

\titleformat{\subsection}
{\normalfont \bf}
{}{0pt}{}
\titlespacing{\subsection}
{0em}{-1em}{0em}

% create a special cvlist environment to format the items
\newenvironment{cvlist}{
\begin{list}{}{\leftmargin=3.0em \itemindent=-3.0em}
  \setlength{\itemsep}{0pt}
  \setlength{\parskip}{0em}
  \setlength{\parsep}{1em}
  \setlength{\parindent}{0em}}
{\vspace{1em}
\end{list}}

% set the default indent to nothing
\setlength{\parindent}{0em}

\begin{document}

% Page layout
\pagestyle{fancy}
\renewcommand{\headrulewidth}{0pt}
\fancyhead{}
\fancyfoot{}
\rhead{{\scriptsize\thepage}}

% git revision control footer 
\rfoot{\texttt{\scriptsize \VCRevision\ on \VCDateTEX}}

% Address and contact block
\begin{minipage}[t]{3in}
 \flushright {\footnotesize Massachusetts Institute of Technology\\ 77 Massachusetts Avenue\\ Cambridge, MA 02139}
  
\end{minipage}
\hfill     
\begin{minipage}[t]{0.0in}
% dummy (needed here)
\end{minipage}
\hfill
\begin{minipage}[t]{1.7in}
  \flushright \footnotesize Phone: \myphone \\ 
  Fax: \myfax \\ 
  {\scriptsize \href{mailto:\myemail}{\myemail}} \\
  {\scriptsize  \href{\myweb}{\myweb}}
\end{minipage}

\medskip

%% Name 
\noindent{\Large {\textsc{\textbf{Benjamin Mako Hill}}}}

\medskip

\section{Education}

\subsection{Massachusetts Institute of Technology}
\begin{cvlist}

\item 2008--Present (ABD). PhD in \emph{Management} and \emph{Media
    Arts and Science} (Interdisciplinary). \\
  Dissertation: Essays on volunteer mobilization in peer production. \\
  Committee: Eric von Hippel, Yochai Benkler, Tom Malone, and
  Mitch Resnick. \\
 GPA: 5.0/5.0 \\
\item 2005--2007. Masters of Science in \emph{Media, Arts, and
    Sciences} from MIT Media Lab.\\ GPA: 5.0/5.0

\end{cvlist}

\subsection{Hampshire College}
\begin{cvlist}
\item 1999--2003. Bachelor of Arts. Major in \emph{Literature,
    Technology and Law}. GPA: N/A
\end{cvlist}

\section{Appointments}

\subsection{Harvard University}
\begin{cvlist}
\item 2011--Present. Fellow, Berkman Center for Internet and Society.
\item 2012--Present. Affiliate, Institute for Quantitative Social Science.
\end{cvlist}

\subsection{Massachusetts Institute of Technology}
\begin{cvlist}
\item 2010--Present. Graduate Affiliate, MIT Center For Collective Intelligence.
\item 2010--Present. Teaching Assistant, MIT Sloan School of Management
  and MIT Program in Systems Design and Management.
\item 2007--2011. Research Fellow, MIT Center For Civic Media.
\item 2007--2008. Senior Researcher, MIT Sloan School of Management.
\item 2005--2007. Research Assistant, \emph{Electronic Publishing}
  and \emph{Computing Culture} Research Groups, MIT Media Lab.
\end{cvlist}

\section{Publications}

\subsection{Refereed Papers}
\begin{cvlist}
\item 2012. Hill, Benjamin Mako and Andrés Monroy-Hernández.
  The remixing dilemma: the trade-off between generativity and
  originality. American Behavioral Scientist. (\emph{Forthcoming})
\item 2011. Monroy-Hernández, Andrés, Benjamin Mako Hill, Jazmin
  Gonzalez-Rivero, and danah boyd. Computers can't give credit: How
  automatic attribution falls short in an online remixing
  community. Proceedings of the Conference on Computer Human
  Interaction (CHI) \emph{(Award: CHI '11 Honorable Mention)}.
\item 2010. Buechley, Leah, and Benjamin Mako Hill. LilyPad in the wild:
  How hardware's long tail is supporting new engineering and design
  communities. Proceedings of the Conference on Design of Interactive
  Systems (DIS). Aarhus, Denmark.
\item 2010. Hill, Benjamin Mako, Andrés Monroy Hernández, and Kristina
  Olson. Responses to remixing on a social media sharing
  website. Pp. 74--81 in Proceedings of the 4th AAAI Conference on
  Weblogs and Social Media (ICWSM). Washington, D.C.
\item 2010. Hill, Benjamin Mako. Revealing errors. In \emph{Error:
    Glitch, Noise, and Jam in New Media Cultures} edited by Mark
  Nunes. Continuum. % (An expanded version of the 2007 journal article.)
\item 2007. Hill, Benjamin Mako. Revealing Errors. \emph{Media/Culture
    Journal} 10 (Feature Article).
\item 2004. Coleman, Gabriella, and Benjamin Mako Hill. 2004. How free
  became open and everything else under the sun. \emph{Media/Culture
    Journal} 7 (Feature Article).
\item 2004. Coleman, Gabriella, and Benjamin Mako Hill. The social
  production of ethics in Debian and free software communities:
  Anthropological lessons for vocational ethics. In \emph{Free/Open
    Source Software Development} edited by Stefan Koch.
\item 2003. Michlmayr, Martin, and Benjamin Mako Hill. Quality and the
  reliance on individuals in free software projects. Pp. 105--109 in
  Proceedings of the 3rd Workshop on Open Source Software Engineering.
\end{cvlist}

% \subsection{Refereed Conference Papers}
% \begin{cvlist}
% \end{cvlist}

% \subsection{Book Chapters}
% \begin{cvlist}
% \end{cvlist}

\subsection{Other Publications}
\begin{cvlist}
\item 2011. Hill, Benjamin Mako. Freedom for Users, Not For Software.
  In \emph{Life of the Commons: Another World Is Possible
  Beyond Market} edited by Silke Helfrich and David Bollier,
  Heinrich-Böll-Stiftung. (Published in German as
  \emph{Commons: Für eine neue Politik Jenseits von Markt und Staat})
\item 2010. Monroy-Hernández, Andrés and Benjamin Mako
  Hill. Cooperation and Attribution in an Online Community of
  Young Creators. In Computer Supported Coopreative Work 2010
  (CSCW '10). (Poster)
\item 2008. Hill, Benjamin Mako. Samir Chopra, Scott
  D. Dexter, Decoding Liberation: The Promise of Free and Open Source
  Software. \emph{Minds and Machines} 18:297-299.
  % \item 2007. [S.M. Thesis] Hill, Benjamin Mako. ``Cooperation in
  %   Parallel: A Tool for Supporting Collaborative Writing in
  %   Diverged Documents.'' Masters Thesis, Massachusetts Institute of
  %   Technology, Program in Media Arts and Sciences. Advised by
  %   Walter Bender, Chris Csikszentmihályi, and Gabriella Coleman.
\item 2005. Hill, Benjamin Mako. Reflections on free
  software past and future. \emph{First Monday} 10.
  % \item 2003 [B.A. Thesis] Hill, Benjamin Mako. ``Literary
  %   Collaboration and Control A Socio-Historic, Technological and
  %   Legal Analysis.'' Undergraduate Thesis, Hampshire
  %   College. Advised by James Miller, James Wald, and David Bollier.

\end{cvlist}

\subsection{Unpublished and Under Review Working Papers}
\begin{cvlist}
\item Hill, Benjamin Mako. Almost Wikipedia: What eight early online
  collaborative encyclopedia projects reveal about the mechanisms of
  collective action.
\item Hill, Benjamin Mako. Causal Effects of a Reputation-Based
  Incentive in an Peer Production Community.
\item Hill, Benjamin Mako and Andrés Monroy-Hernández. Is collaboration
 better for code than for art? Evidence from peer production. (Under
  Review)
\item Hill, Benjamin Mako and Aaron Shaw. Is volunteer labor a
``fixed and finite'' resource? Evidence from peer production.
\item Hill, Benjamin Mako, Aaron Shaw, and Yochai Benkler. Status,
  Social Signalling and Collective Action in a Peer Production
  Community.
\end{cvlist}

\section{Presentations}

\subsection{Selected Invited Research Presentations}
\begin{cvlist}
\item 2012-07-17. Status, Social Signalling and Collective Action in a
  Peer Production Community. Regular Session on Group
  Processes. American Sociological Association Annual Meeting, Denver,
  Colorado.
\item 2012-08-13. Almost Wikipedia: What eight early online
  collaborative encyclopedia projects reveal about the mechanisms of
  collective action. Microsoft Research, Cambridge, Massachusetts.
\item 2012-07-14. Almost Wikipedia: What eight early online
  collaborative encyclopedia projects reveal about the mechanisms of
  collective action.  Wikimania 2012, George Washington University,
  Washington, DC.
\item 2012-07-12. Can can social awards create better wikis? Wikimania
  2012, George Washington University, Washington, DC.
\item 2012-06-29. When peer production works: Learning from failures, to
  improve collaboration. Wikipedia Academy. Freie Universität Berlin ,
  Germany.  (\emph{Keynote address})
\item 2012-04-24. Is volunteer labor a ``fixed and finite'' resource?
  Evidence from peer production. MIT Economic Sociology Working Group.
\item 2011-11-02. Almost Wikipedia: What eight early online
  collaborative encyclopedia projects reveal about the mechanisms of
  collective action. Wikimedia Foundation, San Francisco, California.
\item 2011-10-29. Almost Wikipedia: What eight early online
  collaborative encyclopedia projects reveal about the mechanisms of
  collective action. Conference on Digital Commons, Barcelona, Spain.
\item 2011-10-28. Status, Social Signalling and Collective Action in a
  Peer Production Community. With Aaron Shaw. Laboratory for Social
  Research Seminar, University of California Berkeley.
\item 2011-10-11. Almost Wikipedia: What eight early online collaborative
  encyclopedia projects reveal about the mechanisms of collective
  action. Luncheon Series, Berkman Center for Internet and Society,
  Harvard University.
\item 2011-07-01. Using Social Awards To Build Better Free Software \&
  Free Culture Projects.  With Aaron Shaw. Fórum Internacional
  Software Livre, Porto Alegre, Brasil.
\item 2010-11-17. Almost Wikipedia: What eight early online
  collaborative encyclopedia projects reveal about the mechanisms of
  collecdtive action. MIT Economic Sociology Working Group. % hidden
\item 2010-08-02. ``What the community is remixing:'' The effect of a
  new status-based incentive to collaborate in an online collaborative
  community. MIT Open and User Innovation Workshop.
\item 2010-06-10. Reviewing and challenging socio-political
  approaches in the analysis of open collaboration and collective
  action online. With Mayo Fuster Morell. WikiSym 2010. Gdansk,
  Poland.
\item 2010-06-09. ``What the community is remixing:'' The effect of a
  new status-based incentive to collaborate in an online collaborative
  community. MIT Economic Sociology Working Group. % hidden
\item 2010-04-26. Two empirical analyses of cooperation in
  Scratch. With Andrés Monroy Hernández. Harvard Cooperation Group,
  Berkman Center for Internet and Society.
\item 2009-11-20. The State of FLOSS Research. University of
  Massachusetts Department of Computer Science in Amherst,
  Massachusetts.
\item 2008-10-21. Revealing Errors. Harvard-MIT-Yale Cyberscholar
  Working Group. Harvard University.
\item 2008-04-07. Renaissance Panel: The Roles of Creative Synthesis
  in Innovation. CHI 2008 in Florence, Italy.
\item 2007-11-15. Reflections on Decoding Liberation. Book Launch
  Event for \emph{Decoding Liberation} at Brooklyn College. % hidden
\item 2008-01-22. Clouding Computing and Free and Open Source
  Software. Computing in the Cloud Workshop at the Center for
  Information Technology Policy at Princeton University.
\item 2007-04-27. Reflections on the War on Share. With Elizabeth
  Stark. Media in Transition 5 Conference at MIT.
\item 2007-06-27. Parallel Document Development. User Innovation
  Conference at Copenhagen Business School.
\item 2006-06-02. Defining Moments, Conference on Engaging in Open
  Source (CEOS) organized by the ACM Chapter at Dalhousie University
  in Halifax, Nova Scotia. % hidden

\end{cvlist}

\subsection{Teaching Experience}
\begin{cvlist}
\item 2011--2012. Full Year. Cooperation Group Seminar. Berkman Center
  for Internet and Society, Harvard University.
\item 2010--2012. Spring. Teaching Assistant: How to Develop ``Breakthrough''
  Products and Services. (with Prof. Eric von Hippel). MIT Program in
  Systems Design and Management.
\item 2010--2012. Spring. Teaching Assistant: User-Centric
  Innovations. (with Prof. Eric von Hippel). MIT Sloan School of
  Management.
\item 2008, Fall. Graduate Reading Seminar in Free Software and Open
  Source. MIT Media Lab.
\item 2008, Fall. Seminar in Collective Intelligence. MIT Sloan School
  of Management.
\end{cvlist}

\subsection{Lectures}

I have given dozens of lectures on a variety of subjects. A full
list is available at \url{http://mako.cc/academic}. I have lectured at
MIT, Stanford, Yale, Harvard, and the Evergreen State college. Topics
include:

\begin{cvlist}

\item 2012. User innovation and user communities.
\item 2011--2012. Attracting Participants to user communities.
\item 2008--2012. Introduction to free software and open source.
\item 2008--2012. Hackers: what they do, and why they do it
\item 2011. Designing for cooperation with social incentives.
\item 2008--2010. Revealing errors: what errors can teach us about technology and power.
\item 2010. Free election technologies.
\item 2008. Disasters and Free Software.
\item 2007. Parallel document development: emerging models for
  cooperative writing.

% missing other neil's class?

% \item 2011-11-28. Failure in Free Software and Civic Media. Civic
%   Media, Comparative Media Studies, MIT.

% \item 2011-10-31 Free Software and Free Culture. Difficult Problems
%   in Cyberlaw, Stanford School of Design Stanford Law School.</a>
% \item 2011-09-12. Hackers: What they do, and why they do it. Visiting
%   MBA Class from Vienna University at MIT Sloan School of Management
%   (Philipp Türtscher).

% \item 2009-05-11. Introduction to Free Software and Open Source. MIT
%   Center for Bits and Atom (Neil Gershenfeld). % missing
% \item 2010-11-03. Building Free Election Technologies. MIT Visual Arts
%   Program: Crisis Mapping. % missing
% \item 2010-10-08. Introduction to Free Software and Open Source.  MIT
%   Media Lab: MAS.961 (Design for Empowerement, Leah Buechley). %missing
% \item 2010-05-04. Hackers: What they do, and why they do it. Visiting
%   MBA Class from Vienna University at MIT Sloan School of Management
%   (Philipp Türtscher).
% \item 2010-02-16. Hackers: What they do, and why they do it. MIT Sloan
%   School of Management: 15.356 (Eric von Hippel).
% \item 2010-02-12. Hackers: What they do, and why they do it. MIT
%   Program in Systems Design and Management: 15.969 (Eric von Hippel).

% \item 2009-11-05. Introduction to Free Software and Open Source. MIT
%   Media Lab: Future Craft (Prof. Hiroshi) % missing MAS.921
% \item 2009-02-20. Hackers: What they do, and why they do it. MIT
%   Program in Systems Design and Management: 15.969 (Eric von Hippel).
% \item 2009-02-17. Hackers: What they do, and why they do it. MIT Sloan
%   School of Management: 15.356 (Eric von Hippel).
% \item 2009-11-13. Free Software, Open Source, and Academic Research.
%   MIT Media Lab: MAS.961 (Design for Empowerement, Leah Buechley).
% \item 2009-03-24. Revealing Errors, Yale Law School ISP in New Haven,
%   Connecticut.
% \item 2009-03-24. Introduction to Free/Libre Open Source. For
%   Elizabeth Stark at Yale in New Haven, Connecticut.
% \item 2009-06-16. Free/Libre Open Source Software 101, Knight
%   Foundation News Conference at MIT in Cambridge, Massachusetts.

% \item 2008-10-21. Revealing Errors, Harvard-MIT-Yale Cyberscholar
%   Working Group at the Berkman Center for Internet and Society at
%   Harvard University in Cambridge, Massachusetts
% \item 2008-10-01. Introduction to Free Software and Open Source. MIT
%   Media Lab: Future Craft (Prof. Hiroshi) % missing MAS.921
% \item 2008-04-07. Renaissance Panel: The Roles of Creative Synthesis
%   in Innovation, CHI 2008 in Florence, Italy.
% \item 2008-03-03. Disasters and Free Software, Zones of Emergency
%   series at the Center for Advanced Visual Studies at MIT in
%   Cambridge, Massachusetts.
% \item 2008-01-22. Clouding Computing and Free and Open Source
%   Software, Computing in the Cloud Workshop at the Center for
%   Information Technology Policy at Princeton University in Princeton,
%   NJ.

% \item 2007-11-15. Reflections on Decoding Liberation, Book Launch
%   Event for Decoding Liberation at Brooklyn College.
% \item 2007-06-27. Parallel Document Development, User Innovation
%   Conference at Copenhagen Business School in Copenhagen, Denmark.
% \item 2007-04-27. Reflections on the War on Share, Media in Transition
%   5 (MiT5) at MIT in Cambridge, Massachusetts.
% \item 2006-06-02. Defining Moments, Conference on Engaging in Open
%   Source (CEOS) organized by the ACM Chapter at Dalhousie University
%   in Halifax, Nova Scotia.
% \item 2006-04-28. Information Freedom, MIT's Center for Advanced
%   Visual Studies in Cambridge, Massachusetts.
% \item 2004-08-27 through 2004-09-01. Werkleitz School of Common
%   Property, Halle Volkspark in Halle, Germany.

\end{cvlist}

\section{Other}
\subsection{Other Academic Achievements}

\begin{cvlist}
\item 2010. Passed PhD general examinations in:\\
  (1) technological innovation and entrepreneurship; \\
  (2) organizational sociology; \\
  (3) technology design for creativity and cooperation.
%\item 2003. Although college offered generall does not offer academic
%  achievement awards, undergraduate work featured repeated by
%  in college publications. (GPA N/A)
\item 1998. AP Scholar with Honor.
\item 1998. Washington State, Honors Award recipient.
\end{cvlist}

\subsection{Grants}

\begin{cvlist}
\item 2010--2011. Awarded \$7,500 Educational Research Grant from
  Amazon.

\item 2007. Awarded \$25,000 ``Digital Incubator'' grant from Cisco
  and MTV for academic work on election technology. One of two
  semi-finalists for an additional \$100,000 award.
\end{cvlist}

%\section{Service}
%\begin{cvlist}
%\item Program Committee for WikiSym.
%\item Reviewer for ICIS.
%\end{cvlist}

\section{Work In Industry}

\subsection{Non-Profit Activity}
\begin{cvlist}
\item 2007--Present. \emph{Wikimedia Foundation}, Member, Advisory Board
\item 2008--Present. \emph{Free Software Foundation}, Member, Board of Directors.
\item 2005--Present. \emph{Ubuntu Project}, Core Developer, Member of Community Council.
\item 2005--Present. \emph{One Laptop Per Child}, Member, Advisory Board.
\item 2000--Present. \emph{Debian Project}, Developer, Project Leadership Team (2005-2006).
\item 2005--2008. \emph{Software Freedom International}, Member, Board of Directors.
\item 2005--2008. \emph{Association for Computing Machinery}, Founding Member, Professionals Board.
\item 2002--2006. \emph{Software in the Public Interest}: Vice President and Elected Member, Board of Directors.
\end{cvlist}

\subsection{Selected Employment Experience}
\begin{cvlist}
\item 2004--2005. \emph{Canonical Limited}, Ubuntu project founding team
  member, software engineer, community manager.
\item 2003--2004. \emph{Partecs S.R.L.} (Startup), Chief Technology
  Officer. Rome, Italy.
%\item 2002--2003. System Administrator, University of Massachusetts Dept. of Industrial and Mechanical Engineering.
%\item 2002. Senior Web Application Developer, Hampshire College Career Options Resource Software.
%\item 2001. Lead Web Application Developer, Organizers' Collaborative.
%\item 1999--2003. Assistant to the UNIX System Administrator, Hampshire College.
%\item 2000. Berkman Center for Internet and Society at Harvard Law School: Technical Consultant.
%\item 1999. Technical Consultant and Audio Engineer, Mekana Yesus Recording Studio in Addis Abeba, Ethiopia.
\end{cvlist}


\subsection{Technical Books}
\begin{cvlist}
\item 2006--2011. Hill, Benjamin Mako, Matthew Helmke and Corey
  Burger. The Official Ubuntu Book (Editions published: 2006, 2007,
  2008, 2009, 2010, 2011). New York: Pearson. 2006's best-selling Linux book.
\item 2009, 2011. Rankin, Kyle and Benjamin Mako Hill. The Official Ubuntu
  Server Book. New York: Pearson.
\item 2005. Hill, Benjamin Mako, David B. Harris and Jaldhar
  Vyas. Debian GNU/Linux 3.x Bible. New York: Wiley.
\end{cvlist}

\subsection{Magazine Articles, etc.}
I have published dozens of magazine articles, conference papers, and
other journalistic and non-academic publications. A list can be found
at: \url{http://mako.cc/writing/}

\vspace{2.5em}

\subsection{Examples of Recent Talks}
I have give over a dozen public talks every year since 2002. A
complete list can be found at \url{http://mako.cc/talks}. Some recent
keynote addresses include:

\begin{cvlist}
\item 2011-04-02. When Free Software Isn't Better, Free Software and Linux Days, Istanbul, Turkey.
% \item 2010-03-21. Free Network Services (Panel), The Free Software Foundation's Libre Planet conference at Harvard University in Cambridge, Massachusetts.
\item 2010-01-17. Antifeatures, Linux Conference Australia 2010 in Wellington, New Zealand. % [Keynote]
\item 2009-07-22. The State of Wikimedia Scholarship: 2008-2009, Wikimania 2009 in Buenos Aires, Argentina.
% \item 2009-11-20. Antifeatures, Hampshire College in Amherst, Massachusetts.
% \item 2009-11-18. Antifeatures, NEU ACM Chapter at Northeastern University in Boston, Massachusetts.
% \item 2009-07-22. With Software as a Service, Is Only the Network Luddite Free? (Panel), O'Reilly OSCON in San Jose, California.
% \item 2009-07-22. Antifeatures, O'Reilly OSCON in San Jose, California.
% \item 2009-03-21. Free Network Services, The Free Software Foundation's Libre Planet conference at Harvard University in Cambridge, Massachusetts.
% \item 2009-01-06. Revealing Errors, Razmajena Vjestina skill-sharing meeting at MAMA in Zagreb, Croatia..
\item 2008-10-20. Voting Machinery for the Masses. IEEE Boston Section Society on Social Implications of Technology at MIT Lincoln Labs in Lexington, Massachusetts.
% \item 2008-09-20. Free Software In Your Pocket, Software Freedom Day Boston in Boston, Massachusetts. This talk, delivered with John Sullivan, discussed and shows free software on a variety of mobile devices. This included CHDK, RockBox, and OpenMoko.
% \item 2008-09-11. Advocating Software Freedom by Revealing Errors, O'Reilly Ignite Boston 4 in Boston, Massachusetts.
% \item 2008-07-25. Voting Machinery for the Masses, O'Reilly Open Source Convention (OSCON) in Portland, Oregon. 
\item 2008-07-25. Advocating Software Freedom by Revealing Errors. O'Reilly Open Source Convention (OSCON) in Portland, Oregon. %  (Keynote)
% \item 2008-07-19. Creative Commons Panel, Wikimania 2008 at the Bibliotecha Alexandrina in Alexandria, Egypt.
% \item 2008-07-19. Free Network Services, Wikimania 2008 at the Bibliotecha Alexandrina in Alexandria, Egypt. (Abstract: Link)
% \item 2008-07-17. Zotero for Wikimaniacs, Wikimania 2008 at the Bibliotecha Alexandrina in Alexandria, Egypt.
% \item 2008-06-18. Revealing Errors, Boston Linux Unix in Cambridge, MA
% \item 2008-06-13. Voting Machinery for the Masses, MIT Center for Civic Media's Future of Civic Media Conference at MIT in Cambridge, MA.
% \item 2008-05-29. Voting Machinery for the Masses, O'Reilly's Ignite Boston 3 in Cambridge, MA. (Demo: Link)
\item 2008-05-05. Advancing a Definition of Free Culture. Sun's Community One conference at the Moscone Center in San Francisco, California.
% \item 2008-05-05. Liberating Network Services, Sun's Community One conference at the Moscone Center in San Francisco, California.
\item 2008-04-20. Revealing Errors. Penguicon in Troy, Michigan. % [Keynote]
\item 2008-04-19. Laptop Liberation: One Laptop per Child and Free/Open Source Software. Penguicon in Troy, Michigan. % [Keynote]
% \item 2008-04-13. Revealing Errors, LUG Radio Live USA 2008 in the Moscone Center in San Francisco, CA.
% \item 2008-03-15. Liberating Network Services, FSF Associate Members Meeting at MIT in Cambridge, Massachusetts.
% \item 2008-03-12. Laptop Liberation: One Laptop per Child and Free/Open Source Software, Hampshire College in Amherst, MA.
% \item 2008-03-07. User Innovation in Action. Innovation Lab at MIT Sloan School of Management in Cambridge, MA.
% \item 2008-01-08. Laptop Liberation: One Laptop per Child and Free/Open Source Software, Nara Institute of Technology (NAIST) in  Nara, Japan.
% \item 2007-12-03. Geek Diagnosis from a Diagnosed Geek, G33koSkop lecture series at MAMA in Zagreb, Croatia.
% \item 2007-11-26. Cooperation in Parallel: Lessons from Ubuntu and Debian, Kiberpipa in Ljubljana, Slovenia.
% \item 2007-11-23. Hacker Culture, B92's Cinema Rex in Belgrade, Serbia.
% \item 2007-11-20. Hacker Culture, CK13 in Novi Sad, Serbia.
% \item 2007-11-16. Laptop Liberation: One Laptop per Child and Free/Open Source Software, Cornell University Code Review in Ithaca, NY.
% \item 2007-11-12. Debian Packaging for System Administrators, SIPB Clue Dump at MIT in Cambridge, MA.
\item 2007-10-09. Free Software and Education. K-12 Open Minds Conference in Indianapolis, Indiana. % (Keynote)
% \item 2007-09-15. Free Software and Radical Non-Discrimination, Software Freedom Day 2007 in Boston, MA.
%\item 2007-08-05. Resonant Divergence: Collaboration in Diverged Branches, Wikimania 2007 in Taipei, Taiwan.
% \item 2007-08-03. Freedom's Standard Advanced, Wikimania 2007 in Taipei, Taiwan.
% \item 2007. Debian Derivatives Round Table 2007-06-22, Debconf 7 in Edinburgh, Scotland.
% \item 2007-06-21. Debian: A Force To Be Reckoned With, Debconf 7 in Edinburgh, Scotland.
% \item 2007-06-16. Advancing a Definition of Free Culture, iCommons iSummit in Dubrovnik, Croatia.
% \item 2007-06-07. Examination of Wiki Process, MIT Innovation Lab meeting at the MIT Faculty Club in Cambridge, Massachusetts.
% \item 2007-05-26. Freedom Defined, Annual National Meeting of Free Culture student groups at Harvard University in Cambridge, Massachusetts.
% \item 2007-03-24. Advancing Free Culture, FSF Annual Associate Members Meeting.
% \item 2007-02-16. Contribute To Ubuntu, Google in New York City for The Ubucon NYC 2007.
% \item 2007-02-16. Debian/Ubuntu Packaging Essentials, Google in New York City for The Ubucon NYC 2007.

% \item 2006-09-16. Creative Commons Workshop/Debate, Wizards of OS 4 in Berlin, Germany.
% \item 2006-08-04. Toward a Definition of Freedom, Wikimedia 2006 held at Harvard Law School in Cambridge, Massachusetts.
% \item 2006-04-28. Information Freedom, MIT's Center for Advanced Visual Studies in Cambridge, Massachusetts.

% MAYBE \item 2005-10-28. Software, Freedom, and the World Beyond Computer Programs, Darklight Film Festival's annual symposium in Dublin, Ireland.
% \item 2005-10-19. The Ubuntu Project: Overview and Development Model, Boston Linux Unix meeting at MIT in Cambridge, Massachusetts.
% \item 2005-07-28. To Fork or Not To Fork: Lessons from Ubuntu and Debian, What The Hack near Boxtel in The Netherlands.
% \item 2005-07-06. To Fork or Not To Fork: Lessons from Ubuntu and Debian, Libre Software Meeting in Dijon, France.
% \item 2005-07-05. Broadly Defined Freedom: Radical Nondiscrimination in Free Software, Libre Software Meeting in Dijon, France.
% \item 2005-06-25. To Fork or Not To Fork: Lessons from Ubuntu and Debian, LinuxTag in Karlsruhe, Germany.
% \item 2005-06-24. Financing Volunteer Free Software Projects, LinuxTag in Karlsruhe, Germany. 
% \item 2005-04-24 to 2005-04-30. Ubuntu Down Under, Sydney, Australia.
% \item 2005-04-18. Customizing Debian, Linux Conference Australia 2005 held at Australian National University in Canberra, Australia.
% \item 2005-04-11. Ubuntu Workshop and Q\&A, Northern New Jersey Linux Meet-up in Fort Lee, New Jersey.
% \item 2005-04-10. Ubuntu and Debian: Balancing Forking and Collaboration, Southern Connecticut Open Source User Group in New Haven, Connecticut.
% \item 2005-03-17. Ubuntu and Debian: Balancing Forking and Collaboration, Manizales, Colombia.

% \item 2004-12-14. Customizing Debian, Barcelona at the Grupo de usuarios de Software Libre de Barcelona in Barcelona, Catalonia, Spain.
% \item 2004-11-27. Ubuntu (A GNU/Linux Operating System): Past Present and Future, Congreso GULEV at the World Trade Center in Veracruz, Mexico.
% \item 2004-11-17. Customizing Debian: Fork Yours With Debian GNU/Linux, New York Linux User Group in New York City.
% \item 2004-10-13. Debian and Ubuntu: Philosophy and Technology, New York GNU/Linux Beginners group Gnubies.
% \item 2004-08-27 through 2004-09-01. Werkleitz School of Common Property, Halle Volkspark in Halle, Germany.
% \item 2004-06-04. Financing Volunteer Free Software Projects: Problems and Strategies, Fifth International Free Software Forum in Porto Alegre, Brazil.
% \item 2004-05-30. Software in the Public Interest, Inc. Workshop, Debconf4 in Porto Alegre, Brazil.
% \item 2004-05-31. Custom Debian Distribution are the Ultimate Last Step to Total World Dominations, Debconf4 in Porto Alegre, Brazil.
% \item 2004-06-02. Debian-NP and NP Bagunça Review, Debconf4 in Porto Alegre, Brazil. 
% \item 2004-05-26 to 2004-06-02. Debian-NP Bagunça, Debconf4 in Porto Alegre, Brazil. 
% \item 2004-04-27. Introduction to Debian-NP, LinuxClub in Rome, Italy.
% \item 2004-04-16 - 2004-04-22. Freedom Week (Liberamente - Settimana delle Libertà), Siena, Florence, Milan, Turin, and Rome Italy.
% \item 2004-04-17. Debian-NP: Free Software in Civil Society, Siena, Italy. 
% \item 2004-04-19. Control, Collaboration and Creativity in Literature, University of Milan in Milan, Italy.
% \item 2004-04-21. Participatory Collaboration: The Debian Model, University of Turin in Turin, Italy.
% \item 2004-03-28. Penguin Day, N-TEN's 2004 Nonprofit Technology Conference in Philadelphia, Pennsylvania.

% \item 2003-12-09. Information Politics 101, We Seize! in Geneva, Switzerland.
% \item 2003-12-11. Making Custom Debian Bootable/Live CDs, We Seize! in Geneva, Switzerland.
% \item 2003-12-11. Debian-NP Launch and Q\&A, We Seize! in Geneva, Switzerland.
% \item 2003-11-06. Making the Case for Free/Open Source Software in Non-Profit Organizations, NTEN in Boston, Massachusetts.
% \item 2003-09-09. Digital Standards and the Public Domain: Consequences and Current Strategies for an Independent Public Sphere, Ars Electronica Festival 2003 in Linz, Austria.
% \item 2003-08-26 through 2003-09-06. Summer Source: Software Camp for NGOs, Vis, Croatia.
% \item 2003-07-18. Software in the Public Interest, Inc. Workshop, Debconf 3 in Oslo, Norway.
% \item 2003-07-09. Lessons from Libre Software Political and Ethical Practice, Libre Software Meeting in Metz, France.
% \item 2003-06-07. Social Networking and Free Software, Planetwork Conference in San Francisco, California. 
% \item 2003-04-16. The Politics and Technology of Control, Herb Bernstein's New Ways of Knowing Class at Hampshire College in Amherst, Massachsetts.
% \item 2003-05-05. Presentation to Hampshire College School of Cognitive Science, School of Cognitive Science at Hampshire College in Amherst, Massachusetts.


\end{cvlist}

\end{document}
