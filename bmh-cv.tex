\documentclass[10pt]{article}

% include data on fonts

\usepackage{ucs}
\usepackage[utf8x]{inputenc}

\usepackage[T1]{fontenc}
\usepackage{textcomp}
\renewcommand{\rmdefault}{ugm}
\renewcommand{\sfdefault}{phv}
\usepackage[garamond]{mathdesign}

\usepackage[letterpaper,left=1.75in,right=1.1in,top=1.1in,bottom=1.1in]{geometry}

% Metadata -- alter as neded
\def\myemail{makohill@uw.edu}
\def\myweb{http://mako.cc/academic/}

% Git version tracking 
\input{vc}

% Required style files
\usepackage{url,fancyhdr}

% color for the links 
\usepackage[usenames,dvipsnames]{color}

% import and customize urls
\usepackage[usenames,dvipsnames]{color}
\usepackage[breaklinks]{hyperref}

\hypersetup{colorlinks=true, linkcolor=Blue, citecolor=Black, filecolor=Blue,
    urlcolor=Blue, unicode=true}

% customize the titles so that they appear in the right margin
\usepackage{titlesec}


\setlength{\marginparwidth}{4.2in}
\setlength{\titlewidth}{2.2in}

\titleformat{\section}[leftmargin]
{\normalfont 
\sffamily\bfseries\filleft}
{}{0pt}{\color{BrickRed}}
\titlespacing{\section}
{1.0in}{1.5ex plus .1ex minus .2ex}{1pc}

\titleformat{\subsection}
{\normalfont \bf}
{}{0pt}{}
\titlespacing{\subsection}
{0em}{-1em}{0em}

% create a special cvlist environment to format the items
\newenvironment{cvlist}{
\begin{list}{}{\leftmargin=3.0em \itemindent=-3.0em}
  \setlength{\itemsep}{0pt}
  \setlength{\parskip}{0em}
  \setlength{\parsep}{1em}
  \setlength{\parindent}{0em}}
{\vspace{1em}
\end{list}}

% set the default indent to nothing
\setlength{\parindent}{0em}

\begin{document}

% Page layout
\pagestyle{fancy}
\renewcommand{\headrulewidth}{0pt}
\fancyhead{}
\fancyfoot{}
\rhead{{\scriptsize\thepage}}

% git revision control footer 
\rfoot{\texttt{\scriptsize \VCRevision\ on \VCDateTEX}}

% Address and contact block
\begin{minipage}[t]{3in}
 \flushright {\footnotesize University of Washington\\
   Department of Communication\\
   Box 353740, Seattle, WA, 98195}
  
\end{minipage}
\hfill     
\begin{minipage}[t]{0.0in}
% dummy (needed here)
\end{minipage}
\hfill
\begin{minipage}[t]{1.7in}
  \flushright \footnotesize Phone: (+1) 206.409.7191 \\ 
  \href{mailto:\myemail}{\myemail} \\
  \href{\myweb}{\myweb}
\end{minipage}

\vspace{2em}

%% Name 
\noindent{\Large {\textsc{\textbf{Benjamin Mako Hill}}}}

\medskip

\section{Appointments}

\subsection{University of Washington}
\begin{cvlist}
\item 2014--Present. Assistant Professor, Department of Communication.
\item 2014--Present. Affiliate Faculty, Center for Statistics and the
  Social Sciences.
\item 2014--Present. Affiliate Faculty, eScience Institute
\item 2013--Present. Member, Faculty, DUB (Human Computer Interaction Group).
\item 2013--2014. Acting Assistant Professor, Department of Communication.
\end{cvlist}

\subsection{Harvard University}
\begin{cvlist}
\item 2014--Present. Faculty Affiliate, Berkman Center for Internet and Society.
\item 2012--Present. Affiliate, Institute for Quantitative Social Science.
\item 2011--2014. Fellow, Berkman Center for Internet and Society.
\end{cvlist}

\subsection{Massachusetts Institute of Technology}
\begin{cvlist}
\item 2010--2013. Graduate Affiliate, MIT Center For Collective Intelligence.
\item 2010--2012. Teaching Assistant, MIT Sloan School of Management and MIT Program in Systems Design and Management.
\item 2007--2011. Research Fellow, MIT Center For Civic Media.
\item 2007--2008. Senior Researcher, MIT Sloan School of Management.
\item 2005--2007. Research Assistant, \emph{Electronic Publishing} and \emph{Computing Culture} Research Groups, MIT Media Lab.
\end{cvlist}

\section{Education}

\subsection{Massachusetts Institute of Technology}
\begin{cvlist}

\item 2008--2013. PhD in \emph{Management} and \emph{Media
    Arts and Science} (Interdepartmental).
  % Committee: Eric von Hippel, Yochai Benkler, Tom Malone, and Mitch Resnick. \\
  [GPA: 5.0/5.0] \\
  General examinations in: (1) technological innovation and entrepreneurship; (2) organizational sociology; (3) technology design for creativity and cooperation.

\item 2005--2007. Masters of Science in \emph{Media Arts and
    Sciences} from MIT Media Lab. [GPA: 5.0/5.0]

\end{cvlist}

\subsection{Hampshire College}
\begin{cvlist}
\item 1999--2003. Bachelor of Arts. Major in \emph{Literature,
    Technology and Law} (Self-designed). [GPA: N/A]
\end{cvlist}

\section{Publications}

\subsection{Refereed Articles}
\begin{cvlist}
\item 2015. Huang, Shih-Wen, Minhyang (Mia) Suh, Benjamin Mako Hill, Gary Hsieh. How Activists are Both Born and Made: An Analysis of Users on Change.org. \emph{Proceedings of the Conference on Computer Human Interaction (CHI)}. ACM Press. (Forthcoming) % Seoul, South Korea
\item 2014. Hill, Benjamin Mako and Aaron Shaw. Consider the Redirect: A Missing Dimension of Wikipedia Research. \emph{Proceedings of the 10th International Symposium on Open Collaboration (OpenSym)}. ACM Press.
\item 2014. Zhang, Haoqi, Andrés Monroy-Hernández, Aaron Shaw, Sean A. Munson, Elizabeth Gerber, Benjamin Mako Hill, Peter Kinnaird, Shelly D. Farnham, Patrick Minder. WeDo: End-To-End Computer Supported Collective Action. \emph{Proceedings of the Eighth International AAAI Conference on Weblogs and Social Media (ICWSM)}. AAAI Press. (Short Paper \& Poster)
\item 2014. Shaw, Aaron, and Benjamin Mako Hill. Laboratories of Oligarchy? How the Iron Law Extends to Peer Production. \emph{Journal of Communication} 64-2. Pp. 215–38.
\item 2013. Hill, Benjamin Mako and Aaron Shaw. The Wikipedia gender gap revisited: Characterizing survey response bias with propensity score estimation. \emph{PLoS ONE} 8-6. Pp. e65782.
\item 2013. Hill, Benjamin Mako and Andrés Monroy-Hernández. The remixing dilemma: the trade-off between generativity and originality. \emph{American Behavioral Scientist} 57-5. Pp. 643--663.
\item 2013. Hill, Benjamin Mako and Andrés Monroy-Hernández. The cost of collaboration for code and art: Evidence from a remixing community. \emph{Proceedings of the ACM Conference on Computer-Supported Cooperative Work (CSCW)}. ACM Press. \emph{Award: Best Paper}
\item 2011. Monroy-Hernández, Andrés, Benjamin Mako Hill, Jazmin Gonzalez-Rivero, and danah boyd. Computers can't give credit: How automatic attribution falls short in an online remixing community. \emph{Proceedings of the Conference on Computer Human Interaction (CHI)}. ACM Press. \emph{Award: Honorable Mention}.
\item 2010. Buechley, Leah, and Benjamin Mako Hill. LilyPad in the wild: How hardware's long tail is supporting new engineering and design communities. \emph{Proceedings of the Conference on Design of Interactive Systems (DIS)}. ACM Press.
\item 2010. Hill, Benjamin Mako, Andrés Monroy Hernández, and Kristina Olson. Responses to remixing on a social media sharing website. \emph{Proceedings of the 4th AAAI Conference on Weblogs and Social Media (ICWSM)}. Pp. 74--81. AAAI Press. % Washington, D.C.
\item 2007. Hill, Benjamin Mako. Revealing Errors. \emph{Media/Culture Journal} 10 (Feature Article).
\item 2004. Coleman, Gabriella, and Benjamin Mako Hill. 2004. How free became open and everything else under the sun. \emph{Media/Culture Journal} 7 (Feature Article).
\item 2003. Michlmayr, Martin, and Benjamin Mako Hill. Quality and the reliance on individuals in free software projects. \emph{Proceedings of the 3rd Workshop on Open Source Software Engineering (WOSSE)}. Pp. 105--109. IEEE Press.
\end{cvlist}

\subsection{Books Chapters, Invited Articles, and Other Publications}
\begin{cvlist}
\item 2015. [Book Chapter] Benkler, Yochai, Aaron Shaw, and Benjamin Mako Hill. Peer Production: A Form of Collective Intelligence. In \emph{Collective Intelligence}, edited by Thomas Malone and Michael Bernstein. MIT Press. (Forthcoming)
\item 2014. [Invited Article] Shaw, Aaron, Haoqi Zhang, Andrés Monroy-Hernández, Sean Munson, Benjamin Mako Hill, Elizabeth Gerber, Peter Kinnaird, and Patrick Minder. Computer Supported Collective Action. \emph{Interactions} 21, no. 2 74–77. IEEE.
\item 2013. [Book Chapter] Buechley, Leah, Jennifer Jacobs, Benjamin Mako Hill. Lilypad in the Wild: Technology DIY, E-Textiles, and Gender. In \emph{Textile Messages: Dispatches from the World of E-Textiles and Education}, edited by Leah Buechley, Kylie Peppler, Michael Eisenberg, and Yasmin Kafai. Peter Lang. (Expanded version of 2010 article.)
\item 2012. [Book Chapter] Hill, Benjamin Mako. Freedom for Users, Not For Software. In \emph{Wealth of the Commons: A World Beyond Market and State}, edited by David Bollier and Silke Helfrich, Levellers Press. Published in German as \emph{Commons: Für eine neue Politik Jenseits von Markt und Staat}, Heinrich-Böll-Stiftung.
\item 2010. [Book Chapter] Hill, Benjamin Mako. Revealing errors. In \emph{Error: Glitch, Noise, and Jam in New Media Cultures} edited by Mark Nunes. Continuum. (Expanded version of 2007 article.)
\item 2010. [Poster] Monroy-Hernández, Andrés and Benjamin Mako Hill. Cooperation and Attribution in an Online Community of Young Creators. \emph{Proceedings of the Conference on Computer Supported Cooperative Work (CSCW)}. ACM Press.
\item 2008. [Book Review] Hill, Benjamin Mako. Samir Chopra, Scott D. Dexter, Decoding Liberation: The Promise of Free and Open Source Software. \emph{Minds and Machines} 18:297-299.
\item 2005. [Invited Article] Hill, Benjamin Mako. Reflections on free software past and future. \emph{First Monday} 10.
\item 2004. [Book Chapter] Coleman, Gabriella, and Benjamin Mako Hill. The social production of ethics in Debian and free software communities: Anthropological lessons for vocational ethics. In \emph{Free/Open Source Software Development} edited by Stefan Koch.
\end{cvlist}

% \subsection{Working Papers (Under Review)}
% \begin{cvlist}
% \item Hill, Benjamin Mako. Almost Wikipedia: What eight early online collaborative encyclopedia projects reveal about the mechanisms of collective action.
% \item Hill, Benjamin Mako. Causal Effects of a Reputation-Based Incentive in an Peer Production Community.
% \item Hill, Benjamin Mako, Aaron Shaw, and Yochai Benkler. Status, Social Signalling and Collective Action in a Peer Production Community.
% \item Hill, Benjamin Mako and Aaron Shaw. Is volunteer labor a ``fixed and finite'' resource? Evidence from peer production.
% \end{cvlist}

\section{Selected Presentations}

\subsection{Paper Presentations}

\begin{cvlist}
\item \emph{Laboratories of oligarchy? How the iron law extends to peer production}:
\item 2013-05-08. MIT Economic Sociology Working Group.
\item 2013-04-16. Harvard Cooperation Group, Berkman Center, Harvard.
\item 2013-04-11. Center for Information Technology Policy, Princeton.
\item 2013-03-13. Online Collective Action Working Group, ECPR, Mainz, Germany.
% Aaron 2014. Annual Midwest Political Science Association Conference, Chicago, April.
% Aaron 2013. Computational Social Science Workshop, University of Chicago, November.
% Aaron 2013. Chinese University of Hong Kong, C-Centre, August.
\end{cvlist}

\begin{cvlist}
\item \emph{Status, Social Signaling and Collective Action in a Peer Production Community}:
\item 2012-08-17. American Sociological Association Annual Meeting, Denver, CO.
\item 2011-10-28. Laboratory for Social Research Seminar, UC Berkeley.
\item 2011-06-05. Open and User Innovation Workshop, Vienna, Austria.
\item 2011-05-18. MIT Economic Sociology Working Group.
\end{cvlist}

\begin{cvlist}
\item \emph{Almost Wikipedia}:
\item 2012-08-13. Microsoft Research, Cambridge, MA.
\item 2012-07-14. Wikimania 2012, George Washington University, DC.
\item 2011-11-02. Wikimedia Foundation, San Francisco, California.
\item 2011-12-29. Conference on Digital Commons, Barcelona, Spain.
\item 2011-10-11. Luncheon Series, Berkman Center, Harvard University.
\item 2011-05-20. MIT Center for Collective Intelligence.
\item 2010-11-22. Technological Innovation, Entrepreneurship, and Strategy Seminar, MIT Sloan.
\item 2010-11-17. MIT Economic Sociology Working Group.
\end{cvlist}

\begin{cvlist}
\item \emph{Is volunteer labor a ``fixed and finite'' resource?  Evidence
  from peer production}:
\item 2012-04-24. MIT Economic Sociology Working Group.
\end{cvlist}

\begin{cvlist}
\item \emph{Causal Effects of a Reputation-Based Incentive in an Peer Production Community}:
\item 2010-10-02. Open and User Innovation Workshop. MIT, Cambridge, MA.
\item 2010-06-09. MIT Economic Sociology Working Group, MIT.
\item 2010-04-26. Harvard Cooperation Group, Berkman Center, Harvard.
\end{cvlist}

\begin{cvlist}
\item \emph{Revealing Errors}:
\item 2009-03-24. Yale Law School in New Haven, Connecticut.
\item 2008-10-21. Harvard-MIT-Yale Cyberscholar Working Group, Harvard.
\end{cvlist}

\subsection{Invited Presentations and Panels}

\begin{cvlist}

\item 2014-11-24. Discussion of \emph{The Internet's Own Boy}. Information \& Society Center, Information School, University of Washington. [Panel]
\item 2014-11-19. Understanding Collaborative Creativity in Scratch. Center for Data Science, University of Washington, Tacoma. [Talk]
\item 2014-11-07. Creativity Without Law in Remixing. Conference on ``Creativity Without Law.'' Case Western University School of Law. [Talk]
\item 2014-06-27. Data and Digital Methods BarCamp. ENSCI \& Medialab, Science Po, Paris, France. [Invited Expert Participant]
\item 2014-06-25. Volunteer Mobilization in Peer Production. Medialab, Sciences Po, Paris, France. [Talk]
\item 2014-03-21. Remixing Research and Scratch Data. Scratch Data Summit, MIT Media Lab, Cambridge, Massachusetts. With Andrés Monroy-Hernández.
\item 2014-01-08. Volunteer Mobilization in Peer Production. DUB Seminar, University of Washington. [Talk]
\item 2013-04-10. Failures of Collective Action. School of Cognitive Science, Hampshire College. [Talk]
\item 2012-10-26. Failures of Collective Action. Department of Communication, University of Washington. [Talk]
\item 2012-06-29. When peer production works: Learning from failures, to improve collaboration. Wikipedia Academy 2012, Freie Universität, Berlin. [Keynote]
\item 2012-07-12. Can can social awards create better wikis? Wikimania 2012, George Washington University, Washington, DC. [Talk]
% \item 2011-07-01. Using Social Awards To Build Better Free Software \& Culture Projects.  With Aaron Shaw. Fórum Internacional Software Livre, Porto Alegre, Brasil. [Talk]
\item 2010-06-10. Reviewing and challenging socio-political approaches in the analysis of open collaboration and collective action online. With Mayo Fuster Morell. WikiSym. Gdansk, Poland. [Panel]
\item 2009-11-20. The State of FLOSS Research. University of Massachusetts Department of Computer Science, Amherst, MA. [Talk]
\item 2008-04-07. Renaissance Panel: The Roles of Creative Synthesis in Innovation. CHI 2008, Florence, Italy. [Panel]
\item 2008-01-22. Clouding Computing and Free and Open Source Software. Computing in the Cloud Workshop, Center for Information Technology Policy, Princeton. [Panel]
\item 2007-11-15. Reflections on Decoding Liberation. Book Launch, Brooklyn College. [Talk]
\item 2007-06-27. Parallel Document Development. User Innovation Conference, Copenhagen Business School. [Talk]
\item 2007-04-27. Reflections on the War on Share. With Elizabeth Stark. Media in Transition 5, MIT. [Talk]
\item 2006-06-02. Defining Moments, Conference on Engaging in Open Source (CEOS), Dalhousie University, Halifax, Nova Scotia. [Keynote]
\end{cvlist}

\section{Teaching}

\subsection{Courses}
\begin{cvlist}
\item 2014, Fall. Interpersonal Media (COM482A). UW Department of Communication. (Curriculum covers computer-mediated communication and online communities.) % Evening Degree Program.
\item 2014, Spring. Innovation Communities (COM587B). UW Communication Leadership's ``Masters in Communication in Communities and Networks'' program (professional masters). 
\end{cvlist}

\subsection{Workshops, Seminars, and Directed Readings}
\begin{cvlist}
\item 2013--Present. Co-organizer, Social Computing Reading Group, University of Washington.
\item 2014, Spring \& Fall. Organizer and Lecturer, Community Data Science Workshops, Department of Communication and eScience Institute, University of Washington.
\item 2014, Fall. Directed Reading: Social Computing and Computer Supported Cooperative Work. % sam woolley
\item 2014, Summer. Supervised Independent Study: Quantitative Analyses of Online Collective Action. % mary joyce
\item 2008, Fall. Seminar in Collective Intelligence. MIT Sloan School of Management. Supervised by Thomas Malone.
\item 2011--2013. Cooperation Group Seminar. Berkman Center for Internet and Society, Harvard University.
\end{cvlist}

\subsection{Guest Lectures}

A full list of my lectures is available at \url{http://mako.cc/academic/}. I have lectured at Evergreen State College, Hampshire College, Harvard, Northeastern, Northwestern, MIT, Parsons, Stanford, the University of Washington, and Yale. Topics include:

\begin{cvlist}

\item 2008--2014. Introduction to free software and open source.
\item 2014. From ``free software'' to ``free culture'' and Wikipedia.
\item 2014. Hackers and Innovation: The CHDK Story.
\item 2013--2014. ``Big Data'' research in Communication.
\item 2014. Introduction to Internet research methods.
\item 2012--2013. User innovation and user communities.
\item 2008--2013. Hackers: What they do. Why they do it.
\item 2011--2013. Attracting Participants to user communities.
\item 2013. Wikipedia and organization.
\item 2013. Openness and learning.
\item 2012. Harnessing user innovation with toolkits and user communities.
\item 2011. Designing for cooperation with social incentives.
\item 2008--2010. Revealing errors: What errors can teach us about technology and power.
\item 2010. Free election technologies.
\item 2008. Disasters and Free Software.
\item 2007. Parallel document development: Emerging models for cooperative writing.
\end{cvlist}

\subsection{Students Supervised}

\begin{cvlist}
\item 2014--Present. Samuel Woolley. Committee Member, PhD General Examination Committee, Department of Communication, University of Washington.
\item 2014--Present. Nathan Matthias. Committee Member, PhD General Examination Committee, Program in Media Arts and Sciences, Massachusetts Institute of Technology.
\item 2014--Present. Michael Gilbert. Committee Member, PhD Dissertation Committee, Department of Human Centered Design and Engineering, University of Washington.
\item 2009. Martin Gimpl. MA Thesis Evaluation Committee. Media Lab, University of Art and Design, Helsinki, Finland.
\end{cvlist}

\subsection{Graduate Teaching}
\begin{cvlist}
\item 2010--2012, Spring. Teaching Assistant: How to Develop ``Breakthrough'' Products and Services (15.356). (with Prof. Eric von Hippel). MIT Program in Systems Design and Management.
\item 2010--2012, Spring. Teaching Assistant: User-Centric Innovations (15.969). (with Prof. Eric von Hippel). MIT Sloan School of Management.
\item 2008, Fall. Graduate Seminar in Free Software and Open Source (MAS960). MIT Media Lab. Supervised by Chris Csikszentmihályi.
\end{cvlist}

% missing other neil's class?

% \item 2011-11-28. Failure in Free Software and Civic Media. Civic Media, Comparative Media Studies, MIT.

% \item 2011-10-31 Free Software and Free Culture. Difficult Problems in Cyberlaw, Stanford School of Design Stanford Law School.
% \item 2011-09-12. Hackers: What they do, and why they do it. Visiting MBA Class from Vienna University at MIT Sloan School of Management (Philipp Türtscher).

% \item 2009-05-11. Introduction to Free Software and Open Source. MIT Center for Bits and Atom (Neil Gershenfeld). % missing
% \item 2010-11-03. Building Free Election Technologies. MIT Visual Arts Program: Crisis Mapping. % missing
% \item 2010-10-08. Introduction to Free Software and Open Source.  MIT Media Lab: MAS.961 (Design for Empowerement, Leah Buechley). %missing
% \item 2010-05-04. Hackers: What they do, and why they do it. Visiting MBA Class from Vienna University at MIT Sloan School of Management (Philipp Türtscher).
% \item 2010-02-16. Hackers: What they do, and why they do it. MIT Sloan School of Management: 15.356 (Eric von Hippel).
% \item 2010-02-12. Hackers: What they do, and why they do it. MIT Program in Systems Design and Management: 15.969 (Eric von Hippel).

% \item 2009-11-05. Introduction to Free Software and Open Source. MIT Media Lab: Future Craft (Prof. Hiroshi) % missing MAS.921
% \item 2009-02-20. Hackers: What they do, and why they do it. MIT Program in Systems Design and Management: 15.969 (Eric von Hippel).
% \item 2009-02-17. Hackers: What they do, and why they do it. MIT Sloan School of Management: 15.356 (Eric von Hippel).
% \item 2009-11-13. Free Software, Open Source, and Academic Research. MIT Media Lab: MAS.961 (Design for Empowerement, Leah Buechley).
% \item 2009-03-24. Revealing Errors, Yale Law School ISP in New Haven, Connecticut.
% \item 2009-03-24. Introduction to Free/Libre Open Source. For Elizabeth Stark at Yale in New Haven, Connecticut.
% \item 2009-06-16. Free/Libre Open Source Software 101, Knight Foundation News Conference at MIT in Cambridge, Massachusetts.

% \item 2008-10-21. Revealing Errors, Harvard-MIT-Yale Cyberscholar Working Group at the Berkman Center for Internet and Society at Harvard University in Cambridge, Massachusetts
% \item 2008-10-01. Introduction to Free Software and Open Source. MIT Media Lab: Future Craft (Prof. Hiroshi) % missing MAS.921
% \item 2008-04-07. Renaissance Panel: The Roles of Creative Synthesis in Innovation, CHI 2008 in Florence, Italy.
% \item 2008-03-03. Disasters and Free Software, Zones of Emergency series at the Center for Advanced Visual Studies at MIT in Cambridge, Massachusetts.
% \item 2008-01-22. Clouding Computing and Free and Open Source Software, Computing in the Cloud Workshop at the Center for Information Technology Policy at Princeton University in Princeton, NJ.

% \item 2007-11-15. Reflections on Decoding Liberation, Book Launch Event for Decoding Liberation at Brooklyn College.
% \item 2007-06-27. Parallel Document Development, User Innovation Conference at Copenhagen Business School in Copenhagen, Denmark.
% \item 2007-04-27. Reflections on the War on Share, Media in Transition 5 (MiT5) at MIT in Cambridge, Massachusetts.
% \item 2006-06-02. Defining Moments, Conference on Engaging in Open Source (CEOS) organized by the ACM Chapter at Dalhousie University in Halifax, Nova Scotia.
% \item 2006-04-28. Information Freedom, MIT's Center for Advanced Visual Studies in Cambridge, Massachusetts.
% \item 2004-08-27 through 2004-09-01. Werkleitz School of Common Property, Halle Volkspark in Halle, Germany.

\section{Other Academic}

%\begin{cvlist}
%\item 2003. Although college offered generall does not offer academic achievement awards, undergraduate work featured repeated by in college publications. (GPA N/A)
%\item 1998. AP Scholar with Honor.
%\item 1998. Washington State, Honors Award recipient.
%\end{cvlist}

\subsection{Grants and Awards}

\begin{cvlist}
\item 2014. \emph{National Science Foundation} Grant Award (DRL-1417663) for ``New Pathways into Data Science: Extending the Scratch Programming Language to Enable Youth to Analyze and Visualize Their Own Learning.'' (\$124,374)
\item 2014. \emph{Herbert S. Dordick Award} from the Communication and Technology division of the \emph{International Communication Association} for ``the most outstanding dissertation on communication and technology produced in the preceding year.''
\item 2010--2011. ``Educational Research Grant'' Award from \emph{Amazon}. (\$7,500)
\item 2007. ``Digital Incubator'' grant from \emph{Cisco} and \emph{MTV} for academic work on election technology. (\$25,000; 1 of 2 semi-finalists for \$100,000)
\end{cvlist}

\subsection{Service to Profession}

\begin{cvlist}
\item 2014. Co-organizer. Scratch Data Summit. MIT Media Lab, Cambridge, Massachusetts.
\item 2014. Member, International Advisory Committee. New Research on Digital Societies (NeRDS), Center for Research on Organizations and Workplaces, Kozminski University, Warsaw (Poland).
\item 2014. Member, Program Committee, Open and User Innovation Conference.
\item 2009--2014. Member, Program Committee, OpenSym (formerly WikiSym).
\item 2010, 2012. Program Chair, Open and User Innovation Conference.
\item 2012. Member Program Committee, Wikipedia Academy.
\item Reviewer for a number of journals and conferences in communication, sociology, human computer interaction, and information systems.
% ICIS, CHI, CSCW,
\end{cvlist}

\subsection{University and Departmental Service}
\begin{cvlist}
\item 2014--Present. Colloquium Committee, Department of Communication, University of Washington.
\item 2013--Present. Diversity Committee, Department of Communication, University of Washington.
\item 2014. Member, Search Committee, Senior Lecturer in Communication Leadership, Department of Communication, University of Washington.
\end{cvlist}

\subsection{Theses}

\begin{cvlist}
\item 2013. [Ph.D. Dissertation] Hill, Benjamin Mako. ``Essays on volunteer mobilization in peer production.'' Massachusetts Institute of Technology, Interdepartmental Program in Management and Media Arts and Sciences. Advised by Eric von Hippel, Yochai Benkler, Tom Malone, and Mitch Resnick.
\item 2007. [S.M. Thesis] Hill, Benjamin Mako. ``Cooperation in Parallel: A Tool for Supporting Collaborative Writing in Diverged Documents.'' Massachusetts Institute of Technology, Program in Media Arts and Sciences. Advised by Walter Bender, Chris Csikszentmihályi, and Gabriella Coleman.
\item 2003 [B.A. Thesis] Hill, Benjamin Mako. ``Literary Collaboration and Control A Socio-Historic, Technological and Legal Analysis.'' Hampshire College. Advised by James Miller, James Wald, and David Bollier.
\end{cvlist}

\section{Work In Industry}

\subsection{Non-Profit Activity}
\begin{cvlist}
\item 2008--Present. \emph{Free Software Foundation}, Member, Board of Directors.
\item 2007--Present. \emph{Wikimedia Foundation}, Member, Advisory Board
\item 2014--Present. \emph{Wiki Group Cascadia}, Member, Board of Directors.
\item 2005--Present. \emph{Ubuntu Project}, Core Developer; Member, Community Council (2005--2011).
\item 2000--Present. \emph{Debian Project}, Developer; Project Leadership Team (2005-2006).
\item 2005--2010. \emph{One Laptop Per Child}, Member, Advisory Board. % end date was vague, but 2010 seems safe
\item 2005--2008. \emph{Software Freedom International}, Member, Board of Directors.
\item 2005--2008. \emph{Association for Computing Machinery}, Founding Member, Professionals Board.
\item 2002--2006. \emph{Software in the Public Interest}: Vice President and Elected Member, Board of Directors.
\end{cvlist}

\subsection{Selected Employment Experience}
\begin{cvlist}
\item 2004--2005. \emph{Canonical Limited}, Ubuntu project founding team member, software engineer, community manager.
\item 2003--2004. \emph{Partecs S.R.L.} (Startup), Chief Technology Officer. Rome, Italy.
%\item 2002--2003. System Administrator, University of Massachusetts Dept. of Industrial and Mechanical Engineering.
%\item 2002. Senior Web Application Developer, Hampshire College Career Options Resource Software.
%\item 2001. Lead Web Application Developer, Organizers' Collaborative.
%\item 1999--2003. Assistant to the UNIX System Administrator, Hampshire College.
%\item 2000. Berkman Center for Internet and Society at Harvard Law School: Technical Consultant.
%\item 1999. Technical Consultant and Audio Engineer, Mekana Yesus Recording Studio in Addis Abeba, Ethiopia.
\end{cvlist}

\subsection{Technical Books}
\begin{cvlist}
\item 2006--2013. Hill, Benjamin Mako, Matthew Helmke and Corey Burger. The Official Ubuntu Book (Editions published: 2006, 2007, 2008, 2009, 2010, 2011, 2012). New York: Pearson. 2006's best-selling Linux book.
\item 2009, 2011. Rankin, Kyle and Benjamin Mako Hill. The Official Ubuntu Server Book. New York: Pearson.
\item 2005. Hill, Benjamin Mako, David B. Harris and Jaldhar Vyas. Debian GNU/Linux 3.x Bible. New York: Wiley.
\end{cvlist}

\subsection{Magazine Articles, etc.}

I have published dozens of magazine articles, conference papers, and other journalistic and non-academic publications. A list can be found at: \url{http://mako.cc/writing/}

\vspace{2.5em}

\subsection{Public Talks}

I have give over a dozen public talks every year since 2002. A complete list can be found at \url{http://mako.cc/talks/}. Some recent keynote addresses and major talks include:

\begin{cvlist}
\item 2013-04-20. From Free Software to Free Culture. Free Culture Conference, New York City.
\item 2012-07-23. Open Brands. Awesome Foundation Summit, Cambridge, MA.
\item 2011-04-02. When Free Software Isn't Better. Free Software and Linux Days, Istanbul, Turkey.
% \item 2010-03-21. Free Network Services (Panel), The Free Software Foundation's Libre Planet conference at Harvard University in Cambridge, Massachusetts.
\item 2010-01-17. Antifeatures. Linux Conference Australia 2010 in Wellington, New Zealand. % [Keynote]
\item 2009-07-22. The State of Wikimedia Scholarship: 2008-2009. Wikimania 2009 in Buenos Aires, Argentina.
% \item 2009-11-20. Antifeatures, Hampshire College in Amherst, Massachusetts.
% \item 2009-11-18. Antifeatures, NEU ACM Chapter at Northeastern University in Boston, Massachusetts.
% \item 2009-07-22. With Software as a Service, Is Only the Network Luddite Free? (Panel), O'Reilly OSCON in San Jose, California.
% \item 2009-07-22. Antifeatures, O'Reilly OSCON in San Jose, California.
% \item 2009-03-21. Free Network Services, The Free Software Foundation's Libre Planet conference at Harvard University in Cambridge, Massachusetts.
% \item 2009-01-06. Revealing Errors, Razmajena Vjestina skill-sharing meeting at MAMA in Zagreb, Croatia..
\item 2008-10-20. Voting Machinery for the Masses. IEEE Boston Section Society on Social Implications of Technology at MIT Lincoln Labs in Lexington, Massachusetts.
% \item 2008-09-20. Free Software In Your Pocket, Software Freedom Day Boston in Boston, Massachusetts. This talk, delivered with John Sullivan, discussed and shows free software on a variety of mobile devices. This included CHDK, RockBox, and OpenMoko.
% \item 2008-09-11. Advocating Software Freedom by Revealing Errors, O'Reilly Ignite Boston 4 in Boston, Massachusetts.
% \item 2008-07-25. Voting Machinery for the Masses, O'Reilly Open Source Convention (OSCON) in Portland, Oregon. 
\item 2008-07-25. Advocating Software Freedom by Revealing Errors. O'Reilly Open Source Convention (OSCON) in Portland, Oregon. %  (Keynote)
% \item 2008-07-19. Creative Commons Panel, Wikimania 2008 at the Bibliotecha Alexandrina in Alexandria, Egypt.
% \item 2008-07-19. Free Network Services, Wikimania 2008 at the Bibliotecha Alexandrina in Alexandria, Egypt. (Abstract: Link)
% \item 2008-07-17. Zotero for Wikimaniacs, Wikimania 2008 at the Bibliotecha Alexandrina in Alexandria, Egypt.
% \item 2008-06-18. Revealing Errors, Boston Linux Unix in Cambridge, MA
% \item 2008-06-13. Voting Machinery for the Masses, MIT Center for Civic Media's Future of Civic Media Conference at MIT in Cambridge, MA.
% \item 2008-05-29. Voting Machinery for the Masses, O'Reilly's Ignite Boston 3 in Cambridge, MA. (Demo: Link)
\item 2008-05-05. Advancing a Definition of Free Culture. Sun's Community One conference at the Moscone Center in San Francisco, California.
% \item 2008-05-05. Liberating Network Services, Sun's Community One conference at the Moscone Center in San Francisco, California.
\item 2008-04-20. Revealing Errors. Penguicon in Troy, Michigan. % [Keynote]
\item 2008-04-19. Laptop Liberation: One Laptop per Child and Free/Open Source Software. Penguicon in Troy, Michigan. % [Keynote]
% \item 2008-04-13. Revealing Errors, LUG Radio Live USA 2008 in the Moscone Center in San Francisco, CA.
% \item 2008-03-15. Liberating Network Services, FSF Associate Members Meeting at MIT in Cambridge, Massachusetts.
% \item 2008-03-12. Laptop Liberation: One Laptop per Child and Free/Open Source Software, Hampshire College in Amherst, MA.
% \item 2008-03-07. User Innovation in Action. Innovation Lab at MIT Sloan School of Management in Cambridge, MA.
% \item 2008-01-08. Laptop Liberation: One Laptop per Child and Free/Open Source Software, Nara Institute of Technology (NAIST) in  Nara, Japan.
% \item 2007-12-03. Geek Diagnosis from a Diagnosed Geek, G33koSkop lecture series at MAMA in Zagreb, Croatia.
% \item 2007-11-26. Cooperation in Parallel: Lessons from Ubuntu and Debian, Kiberpipa in Ljubljana, Slovenia.
% \item 2007-11-23. Hacker Culture, B92's Cinema Rex in Belgrade, Serbia.
% \item 2007-11-20. Hacker Culture, CK13 in Novi Sad, Serbia.
% \item 2007-11-16. Laptop Liberation: One Laptop per Child and Free/Open Source Software, Cornell University Code Review in Ithaca, NY.
% \item 2007-11-12. Debian Packaging for System Administrators, SIPB Clue Dump at MIT in Cambridge, MA.
\item 2007-10-09. Free Software and Education. K-12 Open Minds Conference in Indianapolis, Indiana. % (Keynote)
% \item 2007-09-15. Free Software and Radical Non-Discrimination, Software Freedom Day 2007 in Boston, MA.
%\item 2007-08-05. Resonant Divergence: Collaboration in Diverged Branches, Wikimania 2007 in Taipei, Taiwan.
% \item 2007-08-03. Freedom's Standard Advanced, Wikimania 2007 in Taipei, Taiwan.
% \item 2007. Debian Derivatives Round Table 2007-06-22, Debconf 7 in Edinburgh, Scotland.
% \item 2007-06-21. Debian: A Force To Be Reckoned With, Debconf 7 in Edinburgh, Scotland.
% \item 2007-06-16. Advancing a Definition of Free Culture, iCommons iSummit in Dubrovnik, Croatia.
% \item 2007-06-07. Examination of Wiki Process, MIT Innovation Lab meeting at the MIT Faculty Club in Cambridge, Massachusetts.
% \item 2007-05-26. Freedom Defined, Annual National Meeting of Free Culture student groups at Harvard University in Cambridge, Massachusetts.
% \item 2007-03-24. Advancing Free Culture, FSF Annual Associate Members Meeting.
% \item 2007-02-16. Contribute To Ubuntu, Google in New York City for The Ubucon NYC 2007.
% \item 2007-02-16. Debian/Ubuntu Packaging Essentials, Google in New York City for The Ubucon NYC 2007.

% \item 2006-09-16. Creative Commons Workshop/Debate, Wizards of OS 4 in Berlin, Germany.
% \item 2006-08-04. Toward a Definition of Freedom, Wikimedia 2006 held at Harvard Law School in Cambridge, Massachusetts.
% \item 2006-04-28. Information Freedom, MIT's Center for Advanced Visual Studies in Cambridge, Massachusetts.

% MAYBE \item 2005-10-28. Software, Freedom, and the World Beyond Computer Programs, Darklight Film Festival's annual symposium in Dublin, Ireland.
% \item 2005-10-19. The Ubuntu Project: Overview and Development Model, Boston Linux Unix meeting at MIT in Cambridge, Massachusetts.
% \item 2005-07-28. To Fork or Not To Fork: Lessons from Ubuntu and Debian, What The Hack near Boxtel in The Netherlands.
% \item 2005-07-06. To Fork or Not To Fork: Lessons from Ubuntu and Debian, Libre Software Meeting in Dijon, France.
% \item 2005-07-05. Broadly Defined Freedom: Radical Nondiscrimination in Free Software, Libre Software Meeting in Dijon, France.
% \item 2005-06-25. To Fork or Not To Fork: Lessons from Ubuntu and Debian, LinuxTag in Karlsruhe, Germany.
% \item 2005-06-24. Financing Volunteer Free Software Projects, LinuxTag in Karlsruhe, Germany. 
% \item 2005-04-24 to 2005-04-30. Ubuntu Down Under, Sydney, Australia.
% \item 2005-04-18. Customizing Debian, Linux Conference Australia 2005 held at Australian National University in Canberra, Australia.
% \item 2005-04-11. Ubuntu Workshop and Q\&A, Northern New Jersey Linux Meet-up in Fort Lee, New Jersey.
% \item 2005-04-10. Ubuntu and Debian: Balancing Forking and Collaboration, Southern Connecticut Open Source User Group in New Haven, Connecticut.
% \item 2005-03-17. Ubuntu and Debian: Balancing Forking and Collaboration, Manizales, Colombia.

% \item 2004-12-14. Customizing Debian, Barcelona at the Grupo de usuarios de Software Libre de Barcelona in Barcelona, Catalonia, Spain.
% \item 2004-11-27. Ubuntu (A GNU/Linux Operating System): Past Present and Future, Congreso GULEV at the World Trade Center in Veracruz, Mexico.
% \item 2004-11-17. Customizing Debian: Fork Yours With Debian GNU/Linux, New York Linux User Group in New York City.
% \item 2004-10-13. Debian and Ubuntu: Philosophy and Technology, New York GNU/Linux Beginners group Gnubies.
% \item 2004-08-27 through 2004-09-01. Werkleitz School of Common Property, Halle Volkspark in Halle, Germany.
% \item 2004-06-04. Financing Volunteer Free Software Projects: Problems and Strategies, Fifth International Free Software Forum in Porto Alegre, Brazil.
% \item 2004-05-30. Software in the Public Interest, Inc. Workshop, Debconf4 in Porto Alegre, Brazil.
% \item 2004-05-31. Custom Debian Distribution are the Ultimate Last Step to Total World Dominations, Debconf4 in Porto Alegre, Brazil.
% \item 2004-06-02. Debian-NP and NP Bagunça Review, Debconf4 in Porto Alegre, Brazil. 
% \item 2004-05-26 to 2004-06-02. Debian-NP Bagunça, Debconf4 in Porto Alegre, Brazil. 
% \item 2004-04-27. Introduction to Debian-NP, LinuxClub in Rome, Italy.
% \item 2004-04-16 - 2004-04-22. Freedom Week (Liberamente - Settimana delle Libertà), Siena, Florence, Milan, Turin, and Rome Italy.
% \item 2004-04-17. Debian-NP: Free Software in Civil Society, Siena, Italy. 
% \item 2004-04-19. Control, Collaboration and Creativity in Literature, University of Milan in Milan, Italy.
% \item 2004-04-21. Participatory Collaboration: The Debian Model, University of Turin in Turin, Italy.
% \item 2004-03-28. Penguin Day, N-TEN's 2004 Nonprofit Technology Conference in Philadelphia, Pennsylvania.

% \item 2003-12-09. Information Politics 101, We Seize! in Geneva, Switzerland.
% \item 2003-12-11. Making Custom Debian Bootable/Live CDs, We Seize! in Geneva, Switzerland.
% \item 2003-12-11. Debian-NP Launch and Q\&A, We Seize! in Geneva, Switzerland.
% \item 2003-11-06. Making the Case for Free/Open Source Software in Non-Profit Organizations, NTEN in Boston, Massachusetts.
% \item 2003-09-09. Digital Standards and the Public Domain: Consequences and Current Strategies for an Independent Public Sphere, Ars Electronica Festival 2003 in Linz, Austria.
% \item 2003-08-26 through 2003-09-06. Summer Source: Software Camp for NGOs, Vis, Croatia.
% \item 2003-07-18. Software in the Public Interest, Inc. Workshop, Debconf 3 in Oslo, Norway.
% \item 2003-07-09. Lessons from Libre Software Political and Ethical Practice, Libre Software Meeting in Metz, France.
% \item 2003-06-07. Social Networking and Free Software, Planetwork Conference in San Francisco, California. 
% \item 2003-04-16. The Politics and Technology of Control, Herb Bernstein's New Ways of Knowing Class at Hampshire College in Amherst, Massachsetts.
% \item 2003-05-05. Presentation to Hampshire College School of Cognitive Science, School of Cognitive Science at Hampshire College in Amherst, Massachusetts.

\end{cvlist}

\end{document}

%  LocalWords:  mailto eScience Shih Minhyang Suh Hsieh th OpenSym UC
%  LocalWords:  Zhang Haoqi Munson Kinnaird Farnham WeDo Weblogs boyd
%  LocalWords:  generativity Jazmin Rivero danah Gabriella WOSSE Für
%  LocalWords:  Lilypad Kylie Peppler Eisenberg Yasmin Kafai Silke Ph
%  LocalWords:  Helfrich Levellers eine neue Politik Jenseits Markt
%  LocalWords:  und Staat Böll Stiftung Nunes ECPR Mainz Cyberscholar
%  LocalWords:  Own's Internet's Freie Universität Fórum Livre Brasil
%  LocalWords:  Internacional CEOS Scotia Centric Woolley Gimpl DRL
%  LocalWords:  Dordick NeRDS Kozminski Cascadia Buenos Aires BarCamp
%  LocalWords:  ENSCI Medialab
